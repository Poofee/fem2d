% ------------------------------------------------------------------------
%  My Format for a Article

\section{Common Command Set}

There are a number of FEMM-specific Octave that are not associated with any particular problem type.
These functions manipulate the appearance of the main window and
other top-level components like the Lua console and Point
Properties output window.

\begin{itemize}

\item {\tt clearconsole} Clears the output window of the Lua console.

\item {\tt newdocument(doctype)} Creates a new preprocessor document
and opens up a new preprocessor window.  Specify {\tt doctype} to
be {\tt 0} for a magnetics problem, {\tt 1} for an electrostatics
problem, {\tt 2} for a heat flow problem, or {\tt 3} for a current
flow problem. Alternative syntax for this function is {\tt create(doctype)}

\item{\tt hideconsole} Hides the floating Lua console window.

\item{\tt hidepointprops} Hides the floating FEMM Properties display window.

\item \texttt{messagebox('message')} displays the \texttt{'message' }string to the
screen in a pop-up message box.

\item {\tt opendocument('filename')} Opens a document specified by {\tt filename}.

\item \texttt{print(item1,item2,...)} This is standard Lua ``print'' command directed to the
output of the Lua console window. Any number of comma-separated items can be
printed at once via the print command.

\item \texttt{prompt('message')} This function allows a FEMM script to prompt a
user for input. When this command is used, a dialog box pops up with the
\texttt{'message'} string on the title bar of the dialog box. The user can
enter in a single line of input via the dialog box.  Prompt returns the
user's input to Octave and parses it using Octave's {\tt eval}
command.

\item{\tt showconsole} Displays the floating Lua console window.

\item{\tt showpointprops} Displays the floating FEMM Properties
display window.

\item{\tt main\_minimize} minimizes the main FEMM window.

\item{\tt main\_maximize} maximizes the main FEMM window.

\item{\tt main\_restore} restores the main FEMM window from a
 minimized or maximized state.

\item{\tt main\_resize(width,height)} resizes the main FEMM
 window client area to width $\times$ height.

\end{itemize}


\section{Magnetics Preprocessor Command Set}

A number of different commands are available in the preprocessor.

\subsection{Object Add/Remove Commands}

\begin{itemize}

\item{\tt mi\_addnode(x,y)} Add a new node at x,y

\item{\tt mi\_addsegment(x1,y1,x2,y2)} Add a new line segment from node
closest to (x1,y1) to node closest to (x2,y2)

\item{\tt mi\_addblocklabel(x,y)} Add a new block label at (x,y)

\item{\tt mi\_addarc(x1,y1,x2,y2,angle,maxseg)} Add a new arc segment
from the nearest node to (x1,y1) to the nearest node to (x2,y2)
with angle `angle' divided into `maxseg' segments.

\item{\tt mi\_drawline(x1,y1,x2,y2)} Adds nodes at (x1,y1) and
(x2,y2) and adds a line between the nodes.

\item{\tt mi\_drawpolyline([x1,y1;x2,y2'...])} Adds nodes at each
of the specified points and connects them with segments.

\item{\tt mi\_drawpolygon([x1,y1;x2,y2'...])} Adds nodes at each
of the specified points and connects them with segments to form a
closed contour.

\item{\tt mi\_drawarc(x1,y1,x2,y2,angle,maxseg)} Adds nodes at
(x1,y1) and (x2,y2) and adds an arc of the specified angle and
discretization connecting the nodes.

\item{\tt mi\_drawrectangle(x1,y1,x2,y2)} Adds nodes at the
corners of a rectangle defined by the points (x1,y1) and (x2,y2),
then adds segments connecting the corners of the rectangle.

\item{\tt mi\_deleteselected} Delete all selected objects.

\item{\tt mi\_deleteselectednodes} Delete selected nodes.

\item{\tt mi\_deleteselectedlabels} Delete selected block labels.

\item{\tt mi\_deleteselectedsegments} Delete selected segments.

\item{\tt mi\_deleteselectedarcsegments} Delete selects arcs.
\end{itemize}

\subsection{Geometry Selection Commands}

\begin{itemize}
\item{\tt mi\_clearselected} Clear all selected nodes, blocks, segments
and arc segments.

\item{\tt mi\_selectsegment(x,y)} Select the line segment closest to
(x,y)

\item{\tt mi\_selectnode(x,y)} Select the node closest to (x,y).
Returns the coordinates of the selected node.

\item{\tt mi\_selectlabel(x,y)} Select the label closet to (x,y).
Returns the coordinates of the selected label.

\item{\tt mi\_selectarcsegment(x,y)} Select the arc segment closest to
(x,y)

\item{\tt mi\_selectgroup(n)} Select the $n^{th}$ group of nodes, segments, arc
segments and blocklabels. This function will clear all previously selected
elements and leave the editmode in 4 (group)
\end{itemize}

\subsection{Object Labeling Commands}

\begin{itemize}
\item{\tt mi\_setnodeprop('propname',groupno)} Set the selected nodes to
have the nodal property {\tt 'propname'} and group number {\tt groupno}.

\item{\tt mi\_setblockprop('blockname', automesh, meshsize, 'incircuit',
magdir, group, turns)} Set the selected block labels to have the
properties:
\begin{itemize}
\item Block property {\tt 'blockname'}.
\item {\tt automesh}: 0 = mesher defers to mesh size constraint defined in {\tt meshsize},
         1 = mesher automatically chooses the mesh density.
\item {\tt meshsize}: size constraint on the mesh in the block marked by this label.
\item Block is a member of the circuit named {\tt 'incircuit'}
\item The magnetization is directed along an angle in measured in degrees denoted by the parameter
        {\tt magdir}
\item A member of group number {\tt group}
\item The number of turns associated with this label is denoted by {\tt turns}.
\end{itemize}

\item{\tt mi\_setsegmentprop('propname', elementsize, automesh, hide,
group)} Set the selected segments to have:
\begin{itemize}
\item Boundary property {\tt 'propname'}
\item Local element size along segment no greater than {\tt
elementsize}
\item {\tt automesh}:  0 = mesher defers to the element constraint defined by {\tt elementsize},
        1 = mesher automatically chooses mesh size along the selected segments
\item {\tt hide}: 0 =  not hidden in post-processor, 1 == hidden in post processor
\item A member of group number {\tt group}
\end{itemize}

\item{\tt mi\_setarcsegmentprop(maxsegdeg, 'propname', hide, group)} Set the
selected arc segments to:
\begin{itemize}
\item Meshed with elements that span at most {\tt maxsegdeg} degrees per
element
\item Boundary property {\tt 'propname'}
\item {\tt hide}: 0 =  not hidden in post-processor, 1 == hidden in post processor
\item A member of group number {\tt group}
\end{itemize}

\end{itemize}

\subsection{Problem Commands}

\begin{itemize}
\item{\tt mi\_probdef(freq,units,type,precision,depth,minangle,(acsolver))} changes the
problem definition. Set {\tt freq} to the desired frequency in
Hertz.  The {\tt units} parameter specifies the units used for
measuring length in the problem domain.  Valid {\tt 'units'}
entries are {\tt 'inches'}, {\tt 'millimeters'}, {\tt
'centimeters'}, {\tt 'mils'}, {\tt 'meters'}, and {\tt
'micrometers'}. Set the parameter {\tt problemtype} to {\tt 'planar'} for a 2-D
planar problem, or to {\tt 'axi'} for an axisymmetric problem. The
{\tt precision} parameter dictates the precision required by the
solver.  For example, entering {\tt 1E-8} requires the RMS of the
residual to be less than $10^{-8}$.  A fifth parameter, representing the depth of
the problem in the into-the-page direction for 2-D planar problems.  Specify the depth
to be zero for axisymmetric problems. The sixth parameter represents the minimum angle
constraint sent to the mesh generator -- 30 degrees is the usual
choice for this parameter.  The acsolver parameter specifies which solver is to be
used for AC problems: 0 for successive approximation, 1 for Newton.

\item{\tt mi\_analyze(flag)}
runs the magnetics solver.  The {\tt flag} parameter
controls whether the solver window is visible or minimized.  For a
visible window, either specify no value for {\tt flag} or specify
{\tt 0}. For a minimized window, {\tt flag} should be set to {\tt
1}.

\item{\tt mi\_loadsolution} loads and displays the solution corresponding to the
current geometry.

\item {\tt mi\_setfocus('documentname')} Switches the
magnetics input file upon which commands are to act. If
more than one magnetics input file is being edited at a time,
this command can be used to switch between files so that the
mutiple files can be operated upon programmatically. {\tt
'documentname'} should contain the name of the desired document as
it appears on the window's title bar.

\item{\tt mi\_saveas('filename')} saves the file with name {\tt 'filename'}.

\end{itemize}

\subsection{Mesh Commands}

\begin{itemize}
\item{\tt mi\_createmesh} runs triangle to create a mesh. Note that this is not a
necessary precursor of performing an analysis, as {\tt
mi\_analyze} will make sure the mesh is up to date before running
an analysis. The number of elements in the mesh is pushed back onto
the lua stack.

\item{\tt mi\_showmesh} shows the mesh.

\item{\tt mi\_purgemesh} clears the mesh out of both the screen and
memory.
\end{itemize}

\subsection{Editing Commands}

\begin{itemize}
\item{\tt mi\_copyrotate(bx, by, angle, copies )}
        \begin{itemize}
        \item{\tt bx, by} -- base point for rotation
        \item{\tt angle} -- angle by which the selected objects are incrementally
        shifted to make each copy.  {\tt angle} is measured in degrees.
        \item{\tt copies} -- number of copies to be produced from
        the selected objects.
        \end{itemize}

\item{\tt mi\_copyrotate2(bx, by, angle, copies, editaction )}
        \begin{itemize}
        \item{\tt bx, by} -- base point for rotation
        \item{\tt angle} -- angle by which the selected objects are incrementally
        shifted to make each copy.  {\tt angle} is measured in degrees.
        \item{\tt copies} -- number of copies to be produced from
        the selected objects.
        \item{\tt editaction}  0 --nodes, 1 -- lines (segments), 2 --block labels, 3 -- arc
                segments, 4- group
        \end{itemize}

\item{\tt mi\_copytranslate(dx, dy, copies)}
        \begin{itemize}
        \item{\tt dx,dy} -- distance by which the selected objects are incrementally shifted.
        \item{\tt copies} -- number of copies to be produced from the selected objects.
        \end{itemize}

\item{\tt mi\_copytranslate2(dx, dy, copies, editaction)}
        \begin{itemize}
        \item{\tt dx,dy} -- distance by which the selected objects are incrementally shifted.
        \item{\tt copies} -- number of copies to be produced from the selected objects.
        \item{\tt editaction}  0 --nodes, 1 -- lines (segments), 2 --block labels, 3 -- arc
                segments, 4- group
        \end{itemize}

\item{\tt�mi\_createradius(x,�y,�r)}�turns�a�corner�located�at�{\tt�(x,y)}�into�a�curve�of�radius�{\tt�r}.

\item{\tt mi\_moverotate(bx,by,shiftangle)}
        \begin{itemize}
        \item{\tt bx, by} -- base point for rotation
        \item{\tt shiftangle} -- angle in degrees by which the selected objects are rotated.
        \end{itemize}

\item{\tt mi\_moverotate2(bx,by,shiftangle, editaction)}
        \begin{itemize}
        \item{\tt bx, by} -- base point for rotation
        \item{\tt shiftangle} -- angle in degrees by which the selected objects are rotated.
        \item{\tt editaction}  0 --nodes, 1 -- lines (segments), 2 --block labels, 3 -- arc
                segments, 4- group
        \end{itemize}

\item{\tt mi\_movetranslate(dx,dy)}
        \begin{itemize}
        \item{\tt dx,dy} -- distance by which the selected objects are shifted.
        \end{itemize}

\item{\tt mi\_movetranslate2(dx,dy,editaction)}
        \begin{itemize}
        \item{\tt dx,dy} -- distance by which the selected objects are shifted.
        \item{\tt editaction}  0 --nodes, 1 -- lines (segments), 2 --block labels, 3 -- arc
                segments, 4- group
        \end{itemize}

\item{\tt mi\_scale(bx,by,scalefactor)}
        \begin{itemize}
        \item{\tt bx, by} -- base point for scaling
        \item{\tt scalefactor} -- a multiplier that determines how
        much the selected objects are scaled
        \end{itemize}

\item{\tt mi\_scale2(bx,by,scalefactor,editaction)}
        \begin{itemize}
        \item{\tt bx, by} -- base point for scaling
        \item{\tt scalefactor} -- a multiplier that determines how
        much the selected objects are scaled
        \item{\tt editaction}  0 --nodes, 1 -- lines (segments), 2 --block labels, 3 -- arc
                segments, 4- group
        \end{itemize}

\item{\tt mi\_mirror(x1,y1,x2,y2)}
mirror the selected objects about a line passing through the points
{\tt (x1,y1)} and {\tt (x2,y2)}.

\item{\tt mi\_mirror2(x1,y1,x2,y2,editaction)}
mirror the selected objects about a line passing through the points
{\tt (x1,y1)} and {\tt (x2,y2)}. Valid {\tt editaction} entries are
0 for nodes, 1 for lines (segments), 2 for block labels, 3 for arc
segments, and 4 for groups.


\item{\tt mi\_seteditmode(editmode)}
Sets the current editmode to:
        \begin{itemize}
        \item{\tt 'nodes'} - nodes
        \item{\tt 'segments'} - line segments
        \item{\tt 'arcsegments'} - arc segments
        \item{\tt 'blocks'} - block labels
        \item{\tt 'group'} - selected group
        \end{itemize}
This command will affect all subsequent uses of the other editing
commands, if they are used WITHOUT the {\tt editaction} parameter.
\end{itemize}

\subsection{Zoom Commands}

\begin{itemize}
\item{\tt mi\_zoomnatural} zooms to a ``natural'' view with sensible extents.
\item{\tt mi\_zoomout} zooms out by a factor of 50\%.
\item{\tt mi\_zoomin} zoom in by a factor of 200\%.
\item{\tt mi\_zoom(x1,y1,x2,y2)}
Set the display area to be from the bottom left corner specified by
{\tt (x1,y1}) to the top right corner specified by {\tt (x2,y2)}.
\end{itemize}

\subsection{View Commands}

\begin{itemize}
\item{\verb+mi_showgrid+} Show the grid points.
\item{\verb+mi_hidegrid+} Hide the grid points points.
\item{\verb+mi_grid_snap('flag')+}
Setting {\tt flag} to 'on' turns on snap to grid, setting {\tt
flag} to {\tt 'off'} turns off snap to grid.
\item{\verb+mi_setgrid(density,'type')+} Change the grid spacing.  The {\tt density}
parameter specifies the space between grid points, and the {\tt
type} parameter is set to {\tt 'cart'} for cartesian coordinates or
{\tt 'polar'} for polar coordinates.
\item{\tt mi\_refreshview} Redraws the current view.
\item{\tt mi\_minimize} minimizes the active magnetics input view.
\item{\tt mi\_maximize} maximizes the active magnetics input view.
\item{\tt mi\_restore} restores the active magnetics input view from a
 minimized or maximized state.
\item{\tt mi\_resize(width,height)} resizes the active magnetics input
 window client area to width $\times$ height.
%\item{\tt mi\_getview} grabs the currently displayed magnetics input view and imports it into Octave.
\end{itemize}

\subsection{Object Properties}

\begin{itemize}
\item{\tt mi\_addmaterial('matname', mu{\_}x, mu{\_}y, H{\_}c,
J, Cduct, Lam{\_}d, Phi{\_}hmax, lam{\_}fill, LamType, Phi{\_}hx, Phi{\_}hy, nstr, dwire)} adds a
new material with called {\tt 'matname'} with the material
properties:
        \begin{itemize}
        \item{\tt mu{\_}x} Relative permeability in the x- or r-direction.
        \item{\tt mu{\_}y} Relative permeability in the y- or z-direction.
        \item{\tt H{\_}c} Permanent magnet coercivity in
        Amps/Meter.
        \item{\tt J} Applied source current
        density in Amps/mm$^2$.
        \item{\tt Cduct} Electrical conductivity of the material
        in~MS/m.
        \item{\tt Lam{\_}d} Lamination thickness in millimeters.
        \item{\tt Phi\_hmax} Hysteresis lag angle in degrees, used for nonlinear BH curves.
        \item{\tt Lam{\_}fill} Fraction of the volume occupied per lamination that
        is actually filled with iron (Note that this parameter defaults to 1 in the
        {\tt femm} preprocessor dialog box because, by default, iron completely
        fills the volume)
        \item{\tt Lamtype} Set to
                \begin{itemize}
                \item 0 -- Not laminated or laminated in plane
                \item 1 -- laminated x or r
                \item 2 -- laminated y or z
                \item 3 -- magnet wire
                \item 4 -- plain stranded wire
                \item 5 -- Litz wire
                \item 6 -- square wire
                \end{itemize}
        \item{\tt Phi\_hx} Hysteresis lag in degrees in the x-direction for linear problems.
        \item{\tt Phi\_hy} Hysteresis lag in degrees in the y-direction for linear problems.
		\item{\tt nstr} Number of strands in the wire build. Should be 1 for Magnet or Square wire.
		\item{\tt dwire} Diameter of each of the wire's constituent strand in millimeters.
        \end{itemize}
Note that not all properties need be defined -- properties that aren't defined are assigned default
values.


\item{\tt mi\_addbhpoint('blockname',b,h)} Adds a B-H data point the
the material specified by the string {\tt 'blockname'}.  The point to be added
has a flux density of {\tt b} in units of Teslas and a field
intensity of {\tt h} in units of Amps/Meter.

\item{\tt mi\_clearbhpoints('blockname')} Clears all B-H data points
associatied with the material specified by {\tt 'blockname'}.

\item{\tt mi\_addpointprop('pointpropname',a,j)}
adds a new point property of name {\tt 'pointpropname'} with either
a specified potential {\tt a} in units Webers/Meter
or a point current {\tt j} in units of Amps. Set the
unused parameter pairs to 0.

\item{\tt mi\_addboundprop('propname', A0, A1, A2, Phi, Mu, Sig, c0, c1,
BdryFormat)} \\ adds a new boundary property with name {\tt
'propname'}
        \begin{itemize}
        \item For a ``Prescribed A'' type boundary condition, set the {\tt A0,
        A1, A2} and {\tt Phi} parameters as required. Set all other
        parameters to zero.

        \item For a ``Small Skin Depth'' type boundary condtion, set the {\tt Mu}
        to the desired relative permeability and {\tt Sig} to the desired
        conductivity in MS/m.  Set {\tt BdryFormat} to 1 and all other
        parameters to zero.

        \item To obtain a ``Mixed'' type boundary condition, set {\tt C1} and
        {\tt C0} as required and {\tt BdryFormat} to 2.  Set all other
        parameters to zero.

        \item For a ``Strategic dual image'' boundary, set {\tt BdryFormat} to 3
        and set all other parameters to zero.

        \item For a ``Periodic'' boundary condition, set {\tt BdryFormat} to 4 and
        set all other parameters to zero.

        \item For an ``Anti-Perodic'' boundary condition, set {\tt BdryFormat} to
        5 set all other parameters to zero.
        \end{itemize}

\item{\tt mi\_addcircprop('circuitname', i, circuittype)} \\
adds a new circuit property with name {\tt 'circuitname'} with a prescribed current.
The {\tt circuittype} parameter is 0 for a parallel-connected circuit and 1 for a
series-connected circuit.
\item{\tt mi\_deletematerial('materialname')} deletes the material named {\tt
'materialname'}.
\item{\tt mi\_deleteboundprop('propname')} deletes the boundary property named
{\tt 'propname'}.
\item{\tt mi\_deletecircuit('circuitname')} deletes the circuit named {\tt
circuitname}.
\item{\tt mi\_deletepointprop('pointpropname')} deletes the point property named
{\tt 'pointpropname'}
\item{\verb+mi_modifymaterial('BlockName',propnum,value)+} This
function allows for modification of a material's properties without
redefining the entire material ({\em e.g.} so that current can be
modified from run to run).  The material to be modified is
specified by {\tt 'BlockName'}.  The next parameter is the number
of the property to be set. The last number is the value to be
applied to the specified property.  The various properties that can
be modified are listed below:
\begin{center}
\begin{tabular}{lll} \hline
{\tt propnum}& Symbol & Description \\ \hline
 0 & {\tt BlockName} & Name of the material \\
 1 & $\mu_x$ & x (or r) direction relative permeability \\
 2 & $\mu_y$ & y (or z) direction relative permeability \\
 3 & $H_c$   & Coercivity, Amps/Meter \\
 4 & $J$   & Source current density, MA/m$^2$ \\
 5 & $\sigma$ & Electrical conductivity, MS/m \\
 6 & $d_{lam}$  & Lamination thickness, mm \\
 7 & $\phi_{hmax}$ & Hysteresis lag angle for nonlinear problems, degrees \\
 8 & LamFill & Iron fill fraction \\
 9 & LamType & 0 = None/In plane, 1 = parallel to x, 2=parallel to y \\
 10 & $\phi_{hx}$ & Hysteresis lag in x-direction for linear problems, degrees \\
 11 & $\phi_{hy}$ & Hysteresis lag in y-direction for linear problems, degrees \\
 \hline
 \end{tabular}
 \end{center}
\item{\verb+mi_modifyboundprop('BdryName',propnum,value)+}
This function allows for modification of a boundary property. The
BC to be modified is specified by {\tt 'BdryName'}.  The next
parameter is the number of the property to be set. The last number
is the value to be applied to the specified property.  The various
properties that can be modified are listed below:
\begin{center}
\begin{tabular}{lll} \hline
{\tt propnum}& Symbol & Description \\ \hline
 0 & {\tt BdryName} & Name of boundary property \\
 1 & $A_0$ & Prescribed A parameter \\
 2 & $A_1$ & Prescribed A parameter \\
 3 & $A_2$ & Prescribed A parameter \\
 4 & $\phi$ & Prescribed A phase \\
 5 & $\mu$ & Small skin depth relative permeability \\
 6 & $\sigma$ & Small skin depth conductivity, MS/m \\
 7 & $c_0$ & Mixed BC parameter \\
 8 & $c_1$ & Mixed BC parameter \\
 9 & {\tt BdryFormat} & Type of boundary condition: \\
   &                 & 0 = Prescribed A \\
   &                 & 1 = Small skin depth \\
   &                 & 2 = Mixed \\
   &                 & 3 = Strategic Dual Image \\
   &                 & 4 = Periodic \\
   &                 & 5 = Antiperiodic \\ \hline
\end{tabular}
\end{center}
\item{\verb+mi_modifypointprop('PointName',propnum,value)+}
This function allows for modification of a point property. The
point property to be modified is specified by {\tt 'PointName'}.
The next parameter is the number of the property to be set. The
last number is the value to be applied to the specified property.
The various properties that can be modified are listed below:
\begin{center}
\begin{tabular}{lll} \hline
{\tt propnum}& Symbol & Description \\ \hline
 0 & {\tt PointName} & Name of the point property \\
 1 & $A$ & Nodal potential, Weber/Meter \\
 2 & $J$ & Nodal current, Amps\\
 \hline
\end{tabular}
\end{center}
\item{\verb+mi_modifycircprop('CircName',propnum,value)+}
This function allows for modification of a circuit property. The
circuit property to be modified is specified by {\tt 'CircName'}.
The next parameter is the number of the property to be set. The
last number is the value to be applied to the specified property.
The various properties that can be modified are listed below:
\begin{center}
\begin{tabular}{lll} \hline
{\tt propnum}& Symbol & Description \\ \hline
 0 & {\tt CircName} & Name of the circuit property \\
 1 & $i$ & Total current \\
 2 & {\tt CircType} & 0 = Parallel, 1 = Series \\ \hline
 \end{tabular}
 \end{center}
\item{\tt mi\_setcurrent('CircName',i)} sets the current in the
circuit specified by {\tt 'CircName'} to {\tt i}.
\end{itemize}

\subsection{Miscellaneous}
\begin{itemize}
\item{\tt mi\_savebitmap('filename')} saves a bitmapped screenshot of the current
view to the file specified by {\tt 'filename'}.
\item{\tt mi\_savemetafile('filename')} saves a metafile screenshot of the current
view to the file specified by {\tt 'filename'}.
\item{\tt mi\_refreshview} Redraws the current view.
\item{\tt mi\_close} Closes current magnetics preprocessor
document and destroys magnetics preprocessor window.
\item{\tt mi\_shownames(flag)} This function allow the user to display or hide the block label
names on screen.  To hide the block label names, {\tt flag} should be 0.  To display the
names, the parameter should be set to 1.
\item{\tt mi\_readdxf('filename')} This function imports a dxf file specified by {\tt 'filename'}.
\item{\tt mi\_defineouterspace(Zo,Ro,Ri)} defines
an axisymmetric external region to be used in conjuction with the
Kelvin Transformation method of modeling unbounded problems.  The
{\tt Zo} parameter is the z-location of the origin of the outer region,
the {\tt Ro} parameter is the radius of the outer region, and the {\tt
Ri} parameter is the radius of the inner region ({\em i.e.} the region of
interest). In the exterior region, the permeability varies as a function of
distance from the origin of the external region.  These parameters
are necessary to define the permeability variation in the external
region.
\item{\tt mi\_attachouterspace} marks all
selected block labels as members of the external region used for
modeling unbounded axisymmetric problems via the Kelvin
Transformation.
\item{\tt mi\_detachouterspace} undefines all selected block labels
as members of the external region used for modeling unbounded axisymmetric
problems via the Kelvin Transformation.
\end{itemize}

\section{Magnetics Post Processor Command Set}

There are a number of scripting commands designed to operate in
the postprocessing environment.

\subsection{Data Extraction Commands}

\begin{itemize}
\item{\tt mo\_lineintegral(type)} Calculate the line integral for the defined contour
\begin{small}\begin{center}
\begin{tabular}{llllll}\hline
 {\tt type} & name & values 1 & values 2 & values 3 & values 4 \\  \hline
 0 & B.n & total B.n & avg B.n & - & -\\
 1 & H.t & total H.t & avg H.t & - & - \\
 2 & Contour length & surface area & - & -\\
 3 & Stress Tensor Force & DC r/x force & DC y/z force & $2\times$ r/x force & $2\times$ y/z force \\
 4 & Stress Tensor Torque& DC torque & $2\times$ torque & - & - \\
 5 & (B.n)\^{}2 & total (B.n)\^{}2 & avg (B.n)\^{}2 & - & - \\ \hline
\end{tabular}
\end{center}
\end{small}
Returns typically two values. The first
value is the result of the integral calculation, and
the second value is the average of the quantity of interest over the contour.
The only exception is integral 3, which evaluates Maxwell's stress
tensor.  This integral can return up to four results. For force
and torque results, the $2\times$ results are only relevant for
problems where $\omega \neq 0$.

\item{\tt mo\_blockintegral(type)}
Calculate a block integral for the selected blocks
\begin{center}
\begin{tabular}{ll} \hline
 Type & Definition \\ \hline
 0 & $A \cdot J$ \\
 1 & A \\
 2 & Magnetic field energy \\
 3 & Hysteresis and/or lamination losses \\
 4 & Resistive losses \\
 5 & Block cross-section area \\
 6 & Total losses \\
 7 & Total current \\
 8 & Integral of $B_x$ (or $B_r$) over block \\
 9 & Integral of $B_y$ (or $B_z$) over block \\
 10 & Block volume \\
 11 & x (or r) part of steady-state Lorentz force \\
 12 & y (or z) part of steady-state Lorentz force \\
 13 & x (or r) part of $2\times$ Lorentz force \\
 14 & y (or z) part of $2\times$ Lorentz force \\
 15 & Steady-state Lorentz torque \\
 16 & $2 \times$ component of Lorentz torque \\
 17 & Magnetic field coenergy \\
 18 & x (or r) part of steady-state weighted stress tensor force \\
 19 & y (or z) part of steady-state weighted stress tensor force \\
 20 & x (or r) part of $2\times$ weighted stress tensor force \\
 21 & y (or z) part of $2\times$ weighted stress tensor force \\
 22 & Steady-state weighted stress tensor torque \\
 23 & $2 \times$ component of weighted stress tensor torque \\
 24 & $R^2$ ({\em i.e.} moment of inertia / density) \\ \hline
 \hline

 \end{tabular}
 \end{center}

\item{\tt mo\_getpointvalues(x,y)}
Get the values associated with the point at (x,y).  The function returns an array whose contents are, in order:
\begin{center}
\begin{tabular}{ll} \hline
Symbol & Definition \\ \hline
 A & Potential A or flux $\phi$ \\
 B1 &  $B_x$ if planar, $B_r$ if axisymmetric \\
 B2  & $B_y$ if planar, $B_z$ if axisymmetric \\
 Sig & conductivity $\sigma$ \\
 E & stored energy density\\
 H1 & $H_x$ if planar, $H_r$ if axisymmetric \\
 H2 &  $H_y$ if planar, $H_z$ if axisymmetric \\
 Je & eddy current density \\
 Js & source current density\\
 Mu1 & $\mu_x$ if planar, $\mu_r$ if axisymmetric \\
 Mu2 & $\mu_y$ if planar, $\mu_z$ if axisymmetric \\
 Pe & Power density dissipated through ohmic losses \\
 Ph & Power density dissipated by hysteresis \\
 ff & Winding fill factor \\ \hline
 \end{tabular}
\end{center}
The following series of functions retrieves smaller subsets of
these results.

\item{\tt mo\_geta(x,y)} Get the potential associated with the point at (x,y).
For planar problems, the reported potential is vector potential
$A$.  For axisymmetric problems, $2 \pi r A$ is reported.

\item{\tt mo\_getb(x,y)} Get the magnetic flux density associated with the point at (x,y).
The return value is a list with two element representing $B_x$ and
$B_y$ for planar problems and $B_r$ and $B_z$ for axisymmetric
problems.

\item{\tt mo\_getconductivity(x,y)} Gets the conductivity associated with the point at (x,y).

\item{\tt mo\_getenergydensity(x,y)} Gets the magnetic field energy density associated with the point at (x,y).

\item{\tt mo\_geth(x,y)} Get the magnetic field intensity associated with the point at (x,y).
The return value is a list with two element representing $H_x$ and
$H_y$ for planar problems and $H_r$ and $H_z$ for axisymmetric
problems.

\item{\tt mo\_getj(x,y)} Get the electric current density associated with the point at (x,y).

\item{\tt mo\_getmu(x,y)} Get the relative magnetic permeability associated with the point at (x,y).
The return value is a list with two element representing $\mu_x$ and
$\mu_y$ for planar problems and $\mu_r$ and $\mu_z$ for axisymmetric
problems.

\item{\tt mo\_getpe(x,y)} Get the ohmic loss density associated with the point at (x,y).

\item{\tt mo\_getph(x,y)} Get the hysteresis/laminated eddy current loss density associated with the point at (x,y).

\item{\tt mo\_getfill(x,y)} Get the winding factor ({\em i.e.} the average fraction of the volume filled with conductor) associated with the point at (x,y).

\item \verb+mo_makeplot(PlotType,NumPoints,Filename,FileFormat)+
Allows Octave access to FEMM's X-Y plot routines.  If only {\tt PlotType} or only {\tt PlotType}
and {\tt NumPoints} are specified, the command is interpreted as a request to plot the
requested plot type to the screen.  If, in addition, the {\tt Filename} parameter is specified,
the plot is instead written to disk to the specified file name as an extended metafile.
If the {\tt FileFormat} parameter is also, the command is instead interpreted as a command to
write the data to disk to the specfied file name, rather than display it to make a
graphical plot.
Valid entries for {\tt PlotType} are:
\begin{center}
\begin{tabular}{ll} \hline
{\tt PlotType} & Definition \\ \hline
0 & Potential \\
1 & $|B|$ \\
2 & $B \cdot n$ \\
3 & $B \cdot t$ \\
4 & $|H|$ \\
5 & $H \cdot n$ \\
6 & $H \cdot t$ \\
7 & $J_{eddy}$ \\
8 & $J_{source}+J_{eddy}$ \\
\hline
\end{tabular}
\end{center}
Valid file formats are
\begin{center}
\begin{tabular}{ll} \hline
{\tt FileFormat} & Definition \\ \hline
0 & Multi-column text with legend \\
1 & Multi-column text with no legend \\
2 & Mathematica-style formatting \\
\hline
\end{tabular}
\end{center}
For example, if one wanted to plot $B \cdot n$ to the screen with 200 points evaluated to
make the graph, the command would be:

\begin{tabular}{l} {\tt mo\_makeplot(2,200)} \end{tabular}

If this plot were to be written to disk as a metafile, the command would be:

\begin{tabular}{l} \verb+mo_makeplot(2,200,'c:\temp\myfile.emf')+ \end{tabular}

To write data instead of a plot to disk, the command would
be of the form:

\begin{tabular}{l} \verb+mo_makeplot(2,200,'c:\temp\myfile.txt',0)+ \end{tabular}

\item \verb+mo_getprobleminfo+
Returns info on problem description.  Returns two values:
\begin{center}
\begin{tabular}{ll} \hline
Return value & Definition \\ \hline 1 &  problem type  \\ 2 &
frequency in Hz \\ \hline
\end{tabular}
\end{center}

\item{\verb+mo_getcircuitproperties('circuit')+}
Used primarily to obtain impedance information associated with
circuit properties.  Properties are returned for the circuit
property named {\tt 'circuit'}. Six values are returned by the
function.  In order, these parameters are:
\begin{itemize}
        \item{\verb+current+} Current carried by the circuit.
        \item{\verb+volts+}  Voltage drop across the circuit in the circuit.
        \item{\verb+flux+} Circuit's flux linkage

\end{itemize}

\end{itemize}

\subsection{Selection Commands}
\begin{itemize}
\item{\tt mo\_seteditmode(mode)} Sets the mode of the postprocessor to
point, contour, or area mode.  Valid entries for {\tt mode} are
{\tt 'point'}, {\tt 'contour'}, and {\tt 'area'}.
\item{\tt mo\_selectblock(x,y)} Select the block that contains point (x,y).
\item{\tt mo\_groupselectblock(n)} Selects all of the blocks that are labeled by block
        labels that are members of group {\tt n}. If no number is specified ({\em i.e.} {\tt mo\_groupselectblock} ),
                all blocks are selected.

\item{\tt mo\_addcontour(x,y)} Adds a contour point at (x,y). If this
is the first point then it starts a contour, if there are existing
points the contour runs from the previous point to this point. The
{\tt mo\_addcontour} command has the same functionality as a
right-button-click contour point addition when the program is
running in interactive mode.
\item{\tt mo\_bendcontour(angle,anglestep)} Replaces the straight line
formed by the last two points in the contour by an arc that spans {\tt angle}
degrees.  The arc is actually composed of many straight lines, each
of which is constrained to span no more than {\tt anglestep} degrees.
The {\tt angle} parameter can take on values from -180 to 180 degrees.
The {\tt anglestep} parameter must be greater than zero.  If there are less
than two points defined in the contour, this command is ignored.
\item{\tt mo\_selectpoint(x,y)} Adds a contour point at the closest
input point to (x,y).  If the selected point and a previous
selected points lie at the ends of an arcsegment, a contour is
added that traces along the arcsegment.  The {\tt mo\_selectpoint}
command has the same functionality as the left-button-click contour
point selection when the program is running in interactive mode.
\item{\tt mo\_clearcontour} Clear a prevously defined contour
\item{\tt mo\_clearblock} Clear block selection
\end{itemize}

\subsection{Zoom Commands}
\begin{itemize}
\item{\verb+mo_zoomnatural+} Zoom to the natural boundaries of
the geometry.
\item{\verb+mo_zoomin+} Zoom in one level.
\item{\verb+mo_zoomout+} Zoom out one level.
\item{\tt mo\_zoom(x1,y1,x2,y2)} Zoom to the window defined by lower left corner (x1,y1)
and upper right corner (x2,y2).

\end{itemize}

\subsection{View Commands}
\begin{itemize}
\item{\verb+mo_showmesh+} Show the mesh.
\item{\verb+mo_hidemesh+} Hide the mesh.
\item{\verb+mo_showpoints+} Show the node points from the input geometry.
\item{\verb+mo_hidepoints+} Hide the node points from the input geometry.
\item{\tt mo\_smooth('flag')} This function controls whether or not smoothing
is applied to the $B$ and $H$ fields, which are naturally
piece-wise constant over each element.  Setting {\tt flag} equal to
{\tt 'on'} turns on smoothing, and setting {\tt flag} to {\tt
'off'} turns off smoothing.
\item{\verb+mo_showgrid+} Show the grid points.
\item{\verb+mo_hidegrid+} Hide the grid points points.
\item{\verb+mo_grid_snap('flag')+}
Setting {\tt flag} to 'on' turns on snap to grid, setting {\tt
flag} to {\tt 'off'} turns off snap to grid.
\item{\verb+mo_setgrid(density,'type')+} Change the grid spacing.  The {\tt density}
parameter specifies the space between grid points, and the {\tt
type} parameter is set to {\tt 'cart'} for cartesian coordinates or
{\tt 'polar'} for polar coordinates.
\item{\verb+mo_hidedensityplot+} hides the flux density plot.
\item{\verb+mo_showdensityplot(legend,gscale,upper_B,lower_B,type)+}
Shows the flux density plot with options:
        \begin{itemize}
        \item {\tt legend} Set to {\tt 0} to hide the plot legend or {\tt 1} to show the plot
        legend.
        \item {\tt gscale} Set to {\tt 0} for a colour density plot or {\tt 1} for a grey scale density
        plot.
        \item{\verb+upper_B+} Sets the upper display limit for the density
        plot.
        \item{\verb+lower_B+} Sets the lower display limit for the density
        plot.
        \item{\tt type} Type of density plot to display. Valid
        entries are {\tt 'mag'}, {\tt 'real'}, and {\tt 'imag'} for
        magnitude, real component, and imaginary component of $B$,
        respectively.  Alternatively, current density can be
        displayed by specifying {\tt 'jmag'}, {\tt 'jreal'}, and {\tt 'jimag'} for
        magnitude, real component, and imaginary component of $J$,
        respectively.
        \end{itemize}
\item{\verb+mo_hidecontourplot+} Hides the contour plot.
\item{\verb+mo_showcontourplot(numcontours,lower_A,upper_A,type)+}
shows the $A$ contour plot with options:
        \begin{itemize}
        \item{\tt numcontours} Number of $A$ equipotential lines
        to be plotted.
        \item{\verb+upper_A+} Upper limit for $A$ contours.
        \item{\verb+lower_A+} Lower limit for $A$ contours.
                \item{\verb+type+} Choice of {\tt 'real'}, {\tt 'imag'}, or {\tt 'both'}
                        to show either the real, imaginary of both real and imaginary components of A.
        \end{itemize}

\item{\tt mo\_showvectorplot(type,scalefactor)}
controls the display of vectors denoting the field strength and
direction. The parameters taken are the \texttt{type} of plot,
which should be set to 0 for no vector plot,  1 for the real part of flux density B;
2 for the real part of field intensity H; 3 for the imaginary part of B;
4 for the imaginary part of H; 5 for both the real and imaginary parts of B;
and 6 for both the real and imaginary parts of H. The \texttt{scalefactor}
determines the relative length of the vectors. If the scale is set
to 1, the length of the vectors are chosen so that the highest flux
density corresponds to a vector that is the same length as the
current grid size setting.

\item{\tt mo\_minimize} minimizes the active magnetics output view.
\item{\tt mo\_maximize} maximizes the active magnetics output view.
\item{\tt mo\_restore} restores the active magnetics output view from a
 minimized or maximized state.
\item{\tt mo\_resize(width,height)} resizes the active magnetics output
 window client area to width $\times$ height.
% \item{\tt mo\_getview} grabs the currently displayed magnetics output view and imports it into Octave.
\end{itemize}

\subsection{Miscellaneous}
\begin{itemize}

\item{\tt mo\_close} Closes the current post-processor instance.
\item{\tt mo\_refreshview} Redraws the current view.
\item{\tt mo\_reload} Reloads the solution from disk.
\item{\tt mo\_savebitmap('filename')} saves a bitmapped screen shot of the current
view to the file specified by {\tt 'filename'}.
\item{\tt mo\_savemetafile('filename')} saves a metafile screenshot of the current
view to the file specified by {\tt 'filename'}.
\item{\tt mo\_shownames(flag)} This function allow the user to display or hide the block label
names on screen.  To hide the block label names, {\tt flag} should be 0.  To display the
names, the parameter should be set to 1.
\end{itemize}
