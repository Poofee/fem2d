% ------------------------------------------------------------------------
%  My Format for a Article
% ------------------------------------------------------------------------
\documentclass[12pt]{article}
\oddsidemargin 0.0in
\textwidth 6.5in
\textheight 9in
\topmargin -0.35in
\headheight 0.0in
\hfuzz 10pt

\renewcommand{\thefootnote}{\fnsymbol{footnote}}
\newcommand{\be}{\begin{equation}}
\newcommand{\ee}{\end{equation}}
\newcommand{\Curl}{\nabla \times}
\newcommand{\Div}{\nabla \cdot}

% ------------------------------------------------------------------------
% Definitions so that LaTeX won't throw a
% "Too many unprocessed floats" tantrum
% ------------------------------------------------------------------------
\renewcommand{\topfraction}{.85}
\renewcommand{\bottomfraction}{.7}
\renewcommand{\textfraction}{.15}
\renewcommand{\floatpagefraction}{.66}
\renewcommand{\dbltopfraction}{.66}
\renewcommand{\dblfloatpagefraction}{.66}
\setcounter{topnumber}{9}
\setcounter{bottomnumber}{9}
\setcounter{totalnumber}{20}
\setcounter{dbltopnumber}{9}

% ------------------------------------------------------------------------
% Packages to use for a dvips document:
% ------------------------------------------------------------------------
\usepackage{pslatex}
\usepackage[dvips]{graphicx}
\usepackage[dvips]{hyperref}
\begin{document}

% ------------------------------------------------------------------------
% Title Page:
% ------------------------------------------------------------------------
\thispagestyle{empty}

\vspace*{2.5in}

\vspace*{0.5in}
\begin{center}
{\LARGE Finite Element Method Magnetics: OctaveFEMM}

\vspace*{16pt} {\Large Version 1.2}

\vspace*{16pt} {\Large User's Manual}

\vspace*{16pt} \today

\vspace*{48pt} David Meeker \\
\href{mailto:dmeeker@ieee.org}{\tt dmeeker@ieee.org} \\
\href{http://femm.foster-miller.com}{\tt http://femm.foster-miller.com} \\
\copyright 2006
\end{center}

\newpage


% ------------------------------------------------------------------------
% Main Text:
% ------------------------------------------------------------------------
\section{Introduction}

OctaveFEMM is a Matlab toolbox that allows for the operation of Finite
Element Method Magnetics (FEMM) via a set of Matlab functions.  The toolbox
works with Octave, a Matlab clone.

When OctaveFEMM starts up a FEMM process, the usual FEMM user interface
is displayed and is fully functional.  The user then has the choice of
accomplishing modeling and analysis tasks either exclusively through
functions implemented by the toolbox, or by a combination of manual and
programmatic operations -- whichever is easiest for the task at hand.

The syntax of the OctaveFEMM toolbox closely mirrors that of FEMM's existing
Lua scripting language interface associated with FEMM v4.2.  However, there
are some differences between the Lua functions and the analogous Octave/Matlab
implementations:
\begin{itemize}
\item All strings are enclosed in single quotes, rather than
double quotes as in Lua.
\item Functions in Lua that have no arguments require a set of
empty parentheses after the function name ({\em e.g.} {\tt
mi\_analyze()}).  In Octave or Matlab, no parentheses should be used needed ({\em e.g.} {\tt
mi\_analyze} with the OctaveFEMM toolbox).
\item Several commands have also been added to OctaveFEMM that have no
analog in Lua.  These commands streamline the drawing of new
geometries with the OctaveFEMM toolbox, as well is the collection
of data from solutions.
\end{itemize}

Perhaps the most remarkable difference between Lua and OctaveFEMM,
however, is due to the matrix-oriented nature of Octave/Matlab.  In just about any
OctaveFEMM function in which it would be desiable to enter an array
of points such that multiple copies of an operation are performed, OctaveFEMM
will correctly interpret the input perform the requested operation
on every element in the array.  In addition, for any function in which
the coordinates of a point are required, that point can be
specified as an array with two elements instead of specifying each
element separately.  In functions that require the specification
of multiple points, those points can be entered as an array of
two-element arrays.

\section{Installation and Startup}

\subsection{Installation for Matlab and Octave 3}
The OctaveFEMM distribution is automatically installed with FEMM 4.2 in the
\verb+mfiles+ subdirectory.  The absolute directory is typically
\verb+c:\Program Files\femm42\mfiles+.  To use OctaveFEMM with Octave or
Matlab, this path needs to be added to the program's search path. To add
this path to the search path, type the following lines at the Matlab or
Octave 3.X.X command prompt:

\begin{verbatim}
	addpath('c:\\progra~1\\femm42\\mfiles');
	savepath;
\end{verbatim}

Alternatively, in Matlab, the interactive {\tt pathtool} command can be used
to add the {\tt mfiles} directory to the search path.

\subsection{Installation for Octave 2.1.50 and Octave 2.1.73}

It is recommended that you use a newer version of Octave that has support
for the {\tt actxserver} function.  If this function exists, Octave will use
ActiveX automation to communication with FEMM.  However, OctaveFEMM can
still be used with versions of Octave ({\em e.g.} Octave 2.1.50 and 2.1.73)
that do not support ActiveX. If {\tt actxserver} is not available, a much
slower interprocess communication mechanism based on temporary files will be
used.

Again, to use OctaveFEMM, the directory where the OctaveFEMM mfiles are located
must be permanently added to the search path.  This directory can be added to
the search path by an appropriate modification of the \verb+.octaverc+ initialization
file. For example, in the
Octave 2.1.50 release, Octave looks for \verb+.octaverc+ in the directory:

\verb+C:\Program Files\GNU Octave 2.1.50\octave_files+

To insert OctaveFEMM into the search path, one can create the {\tt .octaverc file} and
add the line:

\verb+LOADPATH = [ '/cygdrive/c/progra~1/femm42/mfiles//', LOADPATH ];+

to add the directory containing the OctaveFEMM commands to the Octave search path.

\subsection{Startup}
To start an OctaveFEMM session, use the {\tt openfemm} function.
This function starts up a FEMM process to which OctaveFEMM calls
are sent. When done with OctaveFEMM, the FEMM process can be shut
down via the {\tt closefemm} function.

A number of examples that use OctaveFEMM to analyze various problems
are included in the directory \verb+cd c:\Program Files\femm42\examples+

%-----------------------------------------------------------------------------------
%
%  OctaveFEMM Documentation
%
%-----------------------------------------------------------------------------------

% Magnetics-specific functions
\input{magn.tex}

% Electrostatics-specific functions
\input{elec.tex}

% Heat flow-specific functions
\input{heat.tex}

% Current flow-specific functions
\input{cflow.tex}

%-----------------------------------------------------------------------------------

\end{document}
